% Author : Dr. Merlin George, Dept. of CSE, MBCET

\documentclass[10pt,aspectratio=1610]{beamer}

% -------------------- PACKAGES --------------------
\usepackage[utf8]{inputenc}
\usepackage{graphicx}
\usepackage{mathtools}
\usepackage{utopia}
\usetheme{CambridgeUS}
\usecolortheme{dolphin}

\usepackage{tikz}
\usetikzlibrary{positioning,shadows}
\usepackage{algorithm2e}

\usepackage{tcolorbox}
\tcbuselibrary{skins}

\usepackage[comma]{natbib}
\usepackage{hyperref}
\setcitestyle{square}

\beamertemplatenavigationsymbolsempty
\usefonttheme{serif}

% -------------------- COLORS --------------------
\definecolor{myNewColorA}{RGB}{10,50,135}
\definecolor{myNewColorB}{RGB}{25,85,154}
\definecolor{myNewColorC}{RGB}{183,195,230}

\setbeamercolor*{palette primary}{bg=myNewColorC}
\setbeamercolor*{palette secondary}{bg=myNewColorB, fg=white}
\setbeamercolor*{palette tertiary}{bg=myNewColorA, fg=white}
\setbeamercolor*{titlelike}{fg=myNewColorA}
\setbeamercolor*{item}{fg=myNewColorA}

% -------------------- REMOVE TOP BAR --------------------
\setbeamertemplate{headline}{}

% -------------------- FOOTER (TITLE + DATE + SLIDE NO) --------------------
\defbeamertemplate*{footline}{projectfoot}
{
  \leavevmode
  \hbox{
  \begin{beamercolorbox}[wd=.55\paperwidth,ht=2.6ex,dp=1ex,center]{title in head/foot}
    \insertshorttitle
  \end{beamercolorbox}
  \begin{beamercolorbox}[wd=.20\paperwidth,ht=2.6ex,dp=1ex,center]{date in head/foot}
    \insertshortdate
  \end{beamercolorbox}
  \begin{beamercolorbox}[wd=.25\paperwidth,ht=2.6ex,dp=1ex,right]{date in head/foot}
    \insertframenumber{} / \inserttotalframenumber\hspace{0.8em}
  \end{beamercolorbox}}
}
\setbeamertemplate{footline}[projectfoot]

% -------------------- SHADOWED BLUE TITLE BLOCK --------------------
\newtcolorbox{titleblock}{
  enhanced,
  colback=myNewColorA,
  colframe=myNewColorA,
  arc=4mm,
  boxrule=0pt,
  left=14pt,
  right=14pt,
  top=12pt,
  bottom=12pt,
  drop shadow={shadow xshift=2mm, shadow yshift=-2mm, fill=black!35}
}

% -------------------- METADATA SAFE --------------------
\title[Patient-Centric Healthcare Using Blockchain]
{Hybrid Knowledge-Graph-Enhanced
Retrieval-Augmented Generation for Academic
Information Systems}

\subtitle[In Venture]{Zeroth Review}

\author[Group Members]{
\textbf{Group Members}\\[0.3cm]
\begin{tabular}{l}
Julia Mariam John \hfill (B23CS2137)\\
Nirmel B Joseph \hfill (B23CS2148)\\
Rohith NS \hfill (B23CS2156)\\
\end{tabular}\\[0.3cm]
\textbf{Project Guide:} Mr.Praveen J.S\\
Assistant Professor, Dept. of CSE, MBCET \vspace{1 cm} 
}

\date[\today]{\today}

% -------------------- DOCUMENT --------------------
\begin{document}

% ================= TITLE SLIDE =================
\begin{frame}[plain]

\begin{center}
\begin{titleblock}
\centering
{\Large\color{white}\bfseries\inserttitle}\\[0.3cm]
{\normalsize\color{white}\centering\insertsubtitle}
\end{titleblock}
\end{center}

\vspace{0.2cm}

\begin{center}
{\normalsize\insertauthor}
\end{center}

\begin{tikzpicture}[remember picture,overlay]
\node[anchor=south]
at ([yshift=0pt]current page.south)
{\includegraphics[height=3.5cm]{Images/mbcet.png}};
\end{tikzpicture}

\end{frame}

% ================= CONTENTS =================
\begin{frame}{Contents}
\small
\begin{itemize}
\item Abstract
\item Introduction
\item Literature Review
\item Research Gaps
\item Problem Statement
\item Objectives
\item Work Flow Diagram
\item Methodology
\item Technology Used
\item Conclusion
\item References
\end{itemize}
\end{frame}

% ================= ABSTRACT =================
\section[Abstract]{Abstract}
\begin{frame}{Abstract}
The primary challenge in academic information systems is that data is largely unstructured and distributed across heterogeneous webpages. Traditional keyword searches and basic RAG models struggle to provide precise results because they lack a deep understanding of how entities are related. Our proposed solution integrates vector-based retrieval with a knowledge graph to ensure context-aware and explainable information access. Furthermore, this project aligns with SDG 4 and SDG 9 by advancing educational technology and intelligent infrastructure.
\end{frame}

% ================= INTRODUCTION =================
\section[Introduction]{Introduction}
\begin{frame}{Introduction}
\begin{itemize}
\item Growing dependence on online academic information 
\item Unstructured and fragmented data across institutional websites
\item Inaccurate or context-poor results from conventional search systems
\item Need for intelligent, knowledge-driven academic information systems
\end{itemize}
\end{frame}

% ================= LITERATURE REVIEW =================
\section[Literature Review]{Literature Review}
\begin{frame}{Literature Review}
\scriptsize
\begin{tabular}{|c|p{2.5cm}|p{2.5cm}|p{2.5cm}|p{2.5cm}|}
\hline
Sl.No & Title & Methodology & Results & Pros / Cons\\
\hline
1 & Retrieval-Augmented Generation for Knowledge-Intensive NLP   & Dense Vector Retrieval + LLM & Improved factual grounding & 1. + Reduces hallucination
– Limited relational reasoning\\
2 & Knowledge Graph–Enhanced Question Answering & Knowledge Graph Reasoning & Accurate entity-based answers & + Explicit relationship modeling
– Requires structured data\\
3 & Domain-Specific Academic QA Systems & Semantic Search + Rule-Based Filtering & Reliable academic information access & + Domain reliability
– Limited scalability\\
\hline
\end{tabular}
\end{frame}

% ================= RESEARCH GAPS =================
\begin{frame}{Research Gaps Identified}
\begin{itemize}
\item Limited relational reasoning in embedding-based RAG systems 
\item Insufficient integration of knowledge graphs in academic domains
\item Lack of intent-aware hybrid retrieval mechanisms 
\item Absence of academic-domain-specific evaluation benchmarks 
\end{itemize}
\end{frame}

% ================= PROBLEM STATEMENT =================
\section[Problem Statement]{Problem Statement}
\begin{frame}{Problem Statement}
To design an intelligent academic information system that delivers accurate, context-aware institutional knowledge using hybrid knowledge graph–enhanced retrieval-augmented generation.
\end{frame}

% ================= OBJECTIVES =================
\begin{frame}{Objectives}
\begin{enumerate}
    \item To clean and store college data efficiently.
    \item To build a Knowledge Graph that links teachers, courses, and departments.
    \item To create an AI that chooses the best way to find an answer.
    \item To support SDG 4 (Better Education) and SDG 9 (Innovation).
\end{enumerate}
\end{frame}



% ================= DIAGRAMS =================
\section[DIAGRAMS]{Diagrams}

\begin{frame}{Work Flow Diagram}
\centering
\includegraphics[width=0.70\textwidth]{wf.png}\\[0.4cm]
{\centering\small Figure: Work Flow Diagram of Hybrid Knowledge-Graph-Enhanced
Retrieval-Augmented Generation for Academic
Information Systems\par}
\end{frame}



% ================= METHODOLOGY =================
\section[Methodology]{Methodology}
\begin{frame}{Methodology}
\begin{itemize}
   \item Clean the text data and divide it into smaller chunks.
    \item Store the processed text in a Vector Database for quick searching.
    \item Construct a Knowledge Graph representing academic relationships.
    \item Use both retrieved text and graph knowledge to generate accurate answers.
\end{itemize}
\end{frame}

% ================= Technology Used =================
\section[Technology Used]{Technology Used}
\begin{frame}{Technology Used}
\begin{itemize}
\item Vector Database: Stores information for fast text-based searching.

\item Knowledge Graphs: Maps clear links between teachers, courses, and departments.

\item Retrieval-Augmented Generation (RAG): Uses retrieved facts to help the AI answer accurately.

\item Large Language Models (LLM): Generates natural responses based only on college data.

\item Hybrid Retrieval: Switches between text search and graph logic for the best result.

\item Data Processing Tools: Cleans and prepares website content for the system.
\end{itemize}
\end{frame}

% ================= TIMELINE =================
\section[Timeline]{Timeline}

\begin{frame}{Project Timeline}
\centering
\begin{tabular}{|c|p{5cm}|c|c|}
\hline
\textbf{Phase} & \textbf{Activity} & \textbf{Duration}\\
\hline
1 & Literature Review and Problem Definition & 
1 Month\\
\hline
2 & Research and Data Collection & 1 Month \\
\hline
3 & System Development & 2 Month\\
\hline
4 & Integration and LLM Grounding & 1 Month \\
\hline
5 & Evaluation and Final Reporting & 1 Month \\
\hline
\end{tabular}
\end{frame}

% ================= BASE PAPER =================

% ================= CONCLUSION =================
\section[Conclusion]{Conclusion}
\begin{frame}{Conclusion}
\begin{itemize}
\item We built a hybrid system to fix the limits of traditional academic search.

\item  Our AI understands both general text and complex entity relationships.

\item  The system provides accurate, grounded, and easy-to-verify answers.

\item This project supports SDG 4 and SDG 9 by improving educational tools.

\item Future work includes adding live data like exam dates and library status.
\end{itemize}
\end{frame}

% ================= REFERENCES =================

\section{References}

\begin{frame}{References I}
[1] J. Linders and J. M. Tomczak, ``Knowledge graph-extended retrieval augmented generation for question answering,'' \textit{Applied Intelligence}, vol. 55, no. 17, pp. 1102--1118, 2025.\\[2mm]
[2] Z. Li, Z. Wang, W. Wang, K. Hung, H. Xie, and F. L. Wang, ``Retrieval-augmented generation for educational application: A systematic survey,'' \textit{Computers and Education: Artificial Intelligence}, vol. 8, p. 100417, 2025.
\end{frame}


% ================= THANK YOU =================
\begin{frame}
\centering
\Huge \textcolor{myNewColorA}{Any Questions?}
\end{frame}


\begin{frame}
\centering
\Huge \textcolor{myNewColorA}{Thank You}
\end{frame}
\end{document}
