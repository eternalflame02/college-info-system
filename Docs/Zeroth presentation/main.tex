\documentclass[10pt,aspectratio=1610]{beamer}

% -------------------- PACKAGES --------------------
\usepackage[utf8]{inputenc}
\usepackage{graphicx}
\usepackage{mathtools}
\usepackage{utopia}
\usetheme{CambridgeUS}
\usecolortheme{dolphin}

\usepackage{tikz}
\usetikzlibrary{positioning,shadows}
\usepackage{algorithm2e}

\usepackage{tcolorbox}
\tcbuselibrary{skins}

\usepackage[comma]{natbib}
\usepackage{hyperref}
\setcitestyle{square}

\usepackage{tabularx}

\beamertemplatenavigationsymbolsempty
\usefonttheme{serif}

% -------------------- COLORS --------------------
\definecolor{myNewColorA}{RGB}{10,50,135}
\definecolor{myNewColorB}{RGB}{25,85,154}
\definecolor{myNewColorC}{RGB}{183,195,230}

\setbeamercolor*{palette primary}{bg=myNewColorC}
\setbeamercolor*{palette secondary}{bg=myNewColorB, fg=white}
\setbeamercolor*{palette tertiary}{bg=myNewColorA, fg=white}
\setbeamercolor*{titlelike}{fg=myNewColorA}
\setbeamercolor*{item}{fg=myNewColorA}

% -------------------- REMOVE TOP BAR --------------------
\setbeamertemplate{headline}{}

% -------------------- FOOTER --------------------
\defbeamertemplate*{footline}{projectfoot}
{
  \leavevmode
  \hbox{
  \begin{beamercolorbox}[wd=.55\paperwidth,ht=2.6ex,dp=1ex,center]{title in head/foot}
    \insertshorttitle
  \end{beamercolorbox}
  \begin{beamercolorbox}[wd=.20\paperwidth,ht=2.6ex,dp=1ex,center]{date in head/foot}
    \insertshortdate
  \end{beamercolorbox}
  \begin{beamercolorbox}[wd=.25\paperwidth,ht=2.6ex,dp=1ex,right]{date in head/foot}
    \insertframenumber{} / \inserttotalframenumber\hspace{0.8em}
  \end{beamercolorbox}}
}
\setbeamertemplate{footline}[projectfoot]

% -------------------- TITLE BLOCK --------------------
\newtcolorbox{titleblock}{
  enhanced,
  colback=myNewColorA,
  colframe=myNewColorA,
  arc=4mm,
  boxrule=0pt,
  left=14pt,
  right=14pt,
  top=12pt,
  bottom=12pt,
  drop shadow={shadow xshift=2mm, shadow yshift=-2mm, fill=black!35}
}

% -------------------- METADATA --------------------
\title[Hybrid KG-RAG]
{Hybrid Knowledge-Graph-Enhanced\\
Retrieval-Augmented Generation for Academic\\
Information Systems}

\subtitle[Zeroth Review]{Zeroth Review}

\author[Group Members]{
\textbf{Group Members}\\[0.3cm]
\begin{tabular}{l}
Julia Mariam John (B23CS2137)\\
Nirmel B Joseph (B23CS2148)\\
Rohith NS (B23CS2156)\\
\end{tabular}\\[0.3cm]
\textbf{Project Guide:} Mr. Praveen J. S.\\
Assistant Professor, Dept. of CSE, MBCET
}

\date[13 Jan 2026]{13 January 2026}

% -------------------- DOCUMENT --------------------
\begin{document}

% ================= TITLE =================
\begin{frame}[plain]
\begin{center}
\begin{titleblock}
{\Large\color{white}\bfseries\inserttitle}\\[0.3cm]
{\normalsize\color{white}\insertsubtitle}
\end{titleblock}
\end{center}

\vspace{0.5cm}

\begin{center}
{\normalsize\insertauthor}
\end{center}

\vfill

\begin{center}
\includegraphics[height=2.8cm]{Images/mbcet.png}
\end{center}

\end{frame}

% ================= CONTENTS =================
\begin{frame}{Contents}
\small
\begin{itemize}
\item Abstract
\item Introduction
\item Literature Review
\item Research Gaps
\item Problem Statement
\item Objectives
\item Work Flow Diagram
\item Methodology
\item Conclusion
\item References
\end{itemize}
\end{frame}

% ================= ABSTRACT =================
\section{Abstract}
\begin{frame}{Abstract}
\small
Academic data is typically unstructured and fragmented across institutional websites, limiting the effectiveness of keyword-based and embedding-only retrieval systems. Hybrid knowledge-graph-enhanced retrieval-augmented generation improves contextual grounding and relational reasoning by combining semantic vector retrieval with structured knowledge graph inference, supporting SDG~4 and SDG~9 \cite{kg_rag_springer,rag_edu_elsevier}.
\end{frame}

% ================= INTRODUCTION =================
\section{Introduction}
\begin{frame}{Introduction}
\begin{itemize}
\item Growing dependence on online academic information
\item Institutional data scattered across multiple web sources
\item Limited relational reasoning in conventional retrieval systems
\item Need for intelligent, knowledge-driven academic information systems
\end{itemize}
\end{frame}

% ================= LITERATURE REVIEW =================
\section{Literature Review}
\begin{frame}{Literature Review}
\scriptsize
\begin{tabularx}{\textwidth}{|c|X|X|X|X|}
\hline
Sl.No & Title & Methodology & Results & Limitations\\
\hline
1 & Knowledge graph-extended RAG for QA \cite{kg_rag_springer} 
  & KG-guided retrieval + LLM 
  & Improved multi-hop accuracy 
  & Requires structured knowledge graph\\
\hline
2 & RAG for educational applications \cite{rag_edu_elsevier} 
  & Semantic retrieval + LLM 
  & Reduced hallucination 
  & Dependent on retriever quality\\
\hline
\end{tabularx}
\end{frame}

% ================= RESEARCH GAPS =================
\begin{frame}{Research Gaps Identified}
\begin{itemize}
\item Limited use of knowledge graphs in academic RAG systems
\item Lack of hybrid vector-graph retrieval pipelines
\item Insufficient explainability in LLM-based academic QA
\end{itemize}
\end{frame}

% ================= PROBLEM STATEMENT =================
\begin{frame}{Problem Statement}
To develop a hybrid knowledge-graph-enhanced retrieval-augmented generation system for accurate and explainable academic information access.
\end{frame}

% ================= OBJECTIVES =================
\begin{frame}{Objectives}
\begin{enumerate}
\item To preprocess and index institutional text using vector embeddings.
\item To construct a knowledge graph of academic entities for Mar Baselios College of Engineering and Technology.
\item To integrate hybrid retrieval for query answering.
\item To improve factual grounding and relational reasoning.
\end{enumerate}
\end{frame}

% ================= WORKFLOW =================
\section{Diagrams}
\begin{frame}{Work Flow Diagram}
\centering
\includegraphics[width=0.7\textwidth]{Images/wf.png}
\end{frame}

% ================= METHODOLOGY =================
\section{Methodology}
\begin{frame}{Methodology}
\begin{itemize}
\item Data scraping, cleaning, and chunking
\item Vector embedding
\item Knowledge graph construction
\item Hybrid retrieval and LLM-based answer generation
\end{itemize}
\end{frame}

% ================= CONCLUSION =================
\section{Conclusion}
\begin{frame}{Conclusion}
\begin{itemize}
\item Hybrid KG-RAG improves academic information retrieval
\item Supports both semantic and relational queries
\item Provides grounded and explainable responses
\item Aligned with SDG~4 and SDG~9
\end{itemize}
\end{frame}

% ================= REFERENCES =================
\section{References}
\begin{frame}{References}
\scriptsize
\bibliographystyle{ieeetr}
\bibliography{references}
\end{frame}

% ================= THANK YOU =================
\begin{frame}
\centering
\Huge \textcolor{myNewColorA}{Thank You}
\end{frame}

\end{document}
