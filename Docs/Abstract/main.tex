\documentclass[12pt]{article}
\usepackage[a4paper,margin=1in]{geometry}
\usepackage{setspace}

\begin{document}

\begin{center}
    \textbf{\Large Hybrid Retrieval-Based Intelligent College Information Assistant}
    
    \vspace{0.2cm}
    
     \textbf{\large Abstract}
\end{center}

\doublespacing

Students often face difficulty in accessing accurate and relevant academic information due to the 
scattered nature of college websites, where details related to departments, courses, faculty, 
regulations, and notices are distributed across multiple pages and formats. Conventional keyword-based 
search mechanisms fail to capture semantic meaning and relational structure present in such academic data.

The objective of this mini project is to design and develop a hybrid retrieval-based intelligent college 
information assistant that provides reliable and context-aware responses grounded in official college 
website data. Motivated by recent studies on hybrid retrieval and retrieval-augmented generation, the 
proposed system combines semantic vector-based retrieval with structured knowledge graph reasoning to 
enhance accuracy and generalization \cite{luo2023hybrid,zhu2025kg2rag}.

The methodology involves scraping and preprocessing unstructured college website content, which is 
indexed using semantic embeddings in a vector database for efficient similarity-based retrieval. In 
parallel, a lightweight knowledge graph is constructed to model structured academic relationships such 
as faculty–course–department associations. A retrieval routing mechanism selects relevant information 
from either the vector store or the knowledge graph, and the retrieved context is supplied to a large 
language model for response generation. This hybrid approach improves factual grounding, supports 
relational queries, and reduces hallucination compared to purely semantic retrieval methods.

\vspace{0.5cm}

\noindent
\textbf{Sustainable Development Goals (SDGs):}  
This mini project supports SDG 4 (Quality Education) by improving access to academic information and 
SDG 9 (Industry, Innovation and Infrastructure) through the application of intelligent information 
retrieval systems.

\vspace{0.5cm}

\noindent
\textbf{Keywords:} Retrieval-Augmented Generation, Hybrid Retrieval, Knowledge Graph, Semantic Search, 
College Information System

\vspace{0.8cm}

\noindent
\textbf{Group Members:}
\begin{itemize}
    \item Julia Mariam John (B23CS2137)
    \item Nirmel B Joseph (B23CS2148)
    \item Rohith NS (B23CS2156)
\end{itemize}

\noindent
\textbf{Guide:} Mr.Praveen J.S, Assistant Professor, Department of Computer Science and Engineering

\vspace{0.8cm}

\bibliographystyle{ieeetr}
\bibliography{references}

\end{document}
